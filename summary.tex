\documentclass{article}
\usepackage{setspace}
\usepackage[margin=1in]{geometry}

\title{Summary of ``Thermal Hall Effect Measurements using Strontium Titanate Microthermometers.''}
\author{Colin Tinsman}
\date{}

\begin{document}
\doublespacing
\maketitle
Thermal measurements are an important tool for studying new materials and identifying novel physics in condensed matter systems. In a sense, they are the most general experimental methods available to a condensed matter experimentalist: any excitation in a solid will carry energy, and thus heat. This makes them particularly applicable to studying systems with novel excitations that do not carry charge. Such measurements can be quite challenging to make, however. They require accurate and precise readings of the temperature at multiple points on a crystal, which are often quite small, only a few millimeters in any direction. These thermometers must be compatible with the cryogenic environment where the measurement takes place, and may also need to be compatible with intense magnetic fields.

This is especially important when making thermal Hall effect measurements: the thermal analogue of the Hall effect. This effect is often minute, rarely more than 1\% of the total thermal conductivity. It also often requires making measurments in magnetic fields of a few Telsa or more. These fields can interfere with standard methods of thermometry relying on resistive thermometers by way of their magnetoresistance. In order to eliminate the systematic issues with these devices, we have exploited the strongly temperature dependent dielectic permittivty of strontium titanate. This material is not itself a ferroelectric, but sits close to a quantum phase transition to a ferroelectric state. This causes its permittivity to increase rapidly until reaching a maximum a few Kelvin above absolute zero. By making small capacitors using this material as a dielectric, we can use this fact to measure temperature with great precision in a way that is not systematically affected by a magnetic field. This on its own opens up new possibilities for making thermal Hall effect measurements in strong magnetic fields, but since the permittivity saturates below a few Kelvin, the sensitivity of these devices rapidly degrades in this range. However, it has been noted that annealing strontium titanate in an oxygen-18 atmosphere can tune the properties of this material close to this quantum critical point, even making it ferroelectric if enough is incorperated. By moving strontium titanate close to this quantum phase transition without going over, the permittivity can be made to keep increasing down to temperatures below 1 Kelvin, where the mangetoresistance issues of conventional thermometers are most severe. 

In order to test these capacitave thermometers, thermal Hall effect measurements were carried out on bismuth. Despite being an ``old material'', one which has been studied experiementally for over a century, thermal Hall effect measurements had only recently been performed on it up to 3 T. With the strontium titante microthermometers, we were able to conduct measurements up to 10 T and at temperatures down to 40 K. 

\end{document}
