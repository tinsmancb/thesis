\documentclass{article}
\usepackage{setspace}
\usepackage[margin=1in]{geometry}

\title{Summary of ``Thermal Hall Effect Measurements using Strontium Titanate Microthermometers.''}
\author{Colin Tinsman}
\date{}

\begin{document}
\doublespacing
\maketitle
Thermal measurements are an important tool for studying new materials and identifying novel physics in condensed matter systems. In a sense, they are the most general experimental methods available to a condensed matter experimentalist: any excitation in a solid will carry energy, and thus heat. This makes them particularly applicable to studying systems with novel excitations that do not carry charge. Such measurements can be quite challenging to make, however. They require accurate and precise readings of the temperature at multiple points on a crystal, which are often quite small, only a few millimeters in any direction. These thermometers must be compatible with the cryogenic environment where the measurement takes place, and may also need to be compatible with intense magnetic fields.



\end{document}
