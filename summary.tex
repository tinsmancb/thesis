\documentclass{article}
\usepackage{setspace}
\usepackage[margin=1in]{geometry}

\title{Summary of ``Thermal Hall Effect Measurements using Strontium Titanate Microthermometers.''}
\author{Colin Tinsman}
\date{}

\begin{docuument}
Thermal measurements are an important tool for studying new materials and identifying novel physics in condensed matter systems. In a sense, they are the most general experimental methods available: any excitation in a solid will carry energy, and thus heat. This makes them particularly applicable to studying systems with excitations that do not carry charge. Such measurements require accurate and precise readings of the temperature at multiple points on a crystal, which are often only a few millimeters in any direction. This is especially important when making thermal Hall effect measurements, the thermal analogue of the Hall effect. It requires making precise measurements in magnetic fields of a few Tesla or more. These fields can interfere with standard methods of thermometry relying on resistive thermometers. To circumvent this issue, we have exploited the strongly temperature dependent dielectric permittivty of strontium titanate. This material sits close to a quantum phase transition to a ferroelectric state. This causes its permittivity to increase rapidly until reaching a maximum a few Kelvin above absolute zero. By making small capacitors using this material, we can measure temperature with precision in a way that is not systematically affected by a magnetic field. This opens up new possibilities for making thermal measurements, but since the permittivity saturates below a few Kelvin, the sensitivity of these devices rapidly degrades in this range. However, annealing strontium titanate in an oxygen-18 atmosphere can tune the properties of this material close to this quantum critical point. By moving strontium titanate closer to this quantum phase transition, the permittivity can potentially keep increasing down to temperatures below 1 Kelvin. To test these thermometers, thermal Hall effect measurements were carried out on crystalline bismuth. Bismuth is one of the best known semimetals, a material which hosts both electrons and holes, as well as high mobility. This property makes it an important system for benchmarking experimental techniques. With the strontium titante microthermometers, we were able to conduct measurements up to 10 T and at temperatures down to 40 K. A large thermal Hall coefficient is measured in this system, indicative of high mobility carriers. Another important application of these thermometry techniques is towards making measurements of frustrated magnets. These are systems where the geometry of the lattice interferes with magnetic ordering, resulting in a variety of new magnetic phases. These systems are difficult to study experimentally since they often have itinerant excitations which do not carry charge, making thermal Hall effect measurements all the more important. We discuss our measurements on strontium copper borate, in which pairs of spin-1/2 sites are paired up in strongly coupled dimers. The magnetic excitations of this system are mobile triplet states, called triplons. It has been predicted that these triplet bands could have non-trivial topology, making strontium copper borate a bosonic topological insulator, and resulting in a specific thermal Hall effect signal. Experimental measurements in this system fail to find this signal, casting doubt on this theory. However, we observe magnetic field dependent longitudinal thermal conductivity at temperatures below 1 K, where the triplet excitations should be frozen out. There may be some relation to another phenomena found in this material in this temperature range: the formation of spin superlattices, states where spins form ordered arrangements larger than the base unit cell of the material. 

\end{document}
